\chapter*{Abstract}

According to the proponents of The Moral Foundation Theory (MFT), there are
at least 5 sets of innate intuitions used in order to do moral judgments. Each
set corresponds to one foundation or dimension of an individual moral matrix.
Extensive experimental data obtained from questionnaires which aim to measure
the individual moral matrix supports that, depending on the self-declared
political affiliation, individuals will be differently attached to these
foundations.

Liberals will rely mainly on \textit{(a) harm / care}, \textit{(b) fairness /
reciprocity} foundations. Conservatives will rely on those dimensions too, but
not as much. Nevertheless, they will consider equally important the others
three dimensions \textit{(c) in-group loyalty}, \textit{(d) authority /
respect} and \textit{(e) purity / sanctity}.

The question that we will try to address with this work is why there is this
difference of behavior in moral judgments between liberals and conservatives.
To do that we will introduce a Statistical Mechanics model of opinion
dynamics of Bayesian agents. Based on experimental evidences we assume that
there are two phases in the agent life that will define it's moral matrix.
The first phase will mimic the childhood - adolescence moral formation period,
in which every agent acts as a Bayesian adaptive learner.  During the second
phase the Bayesian algorithm is frozen and the agents exchange information
discussing issues distributed around the mean moral of the society.



