\chapter*{Resumo}

Moral e ideologia política estão intrinsecamente relacionados
com aprendizado, tipos de personalidade e estratégias cognitivas de
indivíduos. Usando um modelo de agentes Bayesianos adaptativos e interagentes
tentaremos responder como características do aprendizado moral na infância
e adolescência estão relacionadas à ideologia, traços de personalidade e
estratégias cognitivas. Assumimos que o aprendizado moral do agente pode ser
dividido em duas fases. A fase 1, é uma mímica do aprendizado de pessoas na
infância e adolescência, nessa fase, o modelo se assemelha ao aprendizado
Bayesiano supervisionado, onde a estratégia para lidar com novas informações
muda com a quantidade de informação recebida.  Posteriormente, na fase 2,
agentes com estratégias cognitivas fixas discutem assuntos públicos, com
conteúdo moral, e mudam suas opinião com a motivação de diminuir o custo
psicológico de discordância com seus parceiros sociais.

Comparando as assinaturas estatísticas das opiniões dos agentes na fase 2
com assinaturas similares obtidas através do Questionário dos Fundamentos
Morais, concluímos que nosso modelo apresenta diversas caraterísticas que tem
respaldo experimental. Por exemplo, a quantidade de informação moral julgado
na fase 1 está positivamente correlacionado com o liberalismo. Além disso,
agentes que são estatisticamente identificados como liberais se adaptam
mais rapidamente a mudanças na sociedade. Também constatamos que com o
aumento do parâmetro de nosso modelo denominado por pressão social, agentes
estatisticamente identificados com pessoas liberais passam a ter perfis
estatísticos mais parecidos com os de pessoas conservadores.
