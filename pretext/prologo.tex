
\chapter*{Prólogo}

Toda pesquisa científica tem suas peculiaridades. Em trabalhos
multi-disciplinares existe uma grande dificuldade de sintetizar o conteúdo de
áreas de conhecimentos com histórias e \textit{modus operandi} completamente
distintos. Este trabalho tem o objetivo de descrever o aprendizado moral
usando técnicas matemáticas de Mecânica Estatística e Inferência
Bayesiana. O texto é escrito para um leitor que tenha pouca familiaridade
tanto com os métodos matemáticos quanto com as teorias de ciência social
apresentadas. Para isso, o conteúdo necessário para entender o trabalho
é apresentado em duas partes. Na primeira parte, destacamos as teorias e
experimentos relacionados ao aprendizado moral, dando pouca ênfase
aos detalhes matemáticos de nosso modelo. Esperamos, com isso, que um leitor
com pouco domínio das técnicas matemáticas usadas seja capaz de entender
a essência de nossa modelagem. A segunda parte do texto é uma série de
apêndices, sendo os dois primeiros escritos para que um leitor com um pouco
mais de curiosidade sobre os detalhes matemáticos seja capaz de entender
melhor algumas características do modelo. O terceiro e último apêndice
apresenta uma série de resultados para o modelo matemático em que baseamos
o texto principal, escritos sob o contexto de dinâmicas de opiniões.
