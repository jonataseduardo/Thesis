\chapter{Conclusão}
\begin{epigraphs}
\qitem{
"After several unsuccessful attempts to weld my results together into such
a whole, I realized that I should never succeed. The best that I could
write would never be more than philosophical remarks; my thoughts were soon
crippled if I tried to force them on in any single direction against their
natural inclination.—And this was, of course, connected with the very
nature of the investigation. For this compels us to travel over a wide field
of thought criss-cross in every direction.—"
}
{\---- \textsc{Ludwig Wittgenstein, 1889-1951}}
\end{epigraphs}

O principal resultado apresentado aqui é que o estilo cognitivo do agente
Bayesiano depende da complexidade da interação social na Fase 1, sendo que
o estilo cognitivo induz uma associação estatística a uma filiação política
depois que a sociedade de agente atinge o estado estacionário da Fase 2.

A Fase 1 é uma mímica do aprendizado de pessoas durante a pré-adolescência,
ou da infância até o inicio da fase adulta, já a Fase 2 dois tenta
reproduzir o aprendizado de pessoas na fase adulta. Na Fase 2, a estratégia
cognitiva do agente é cristalizada, ou seja, ela deixa de evoluir de acordo
com a informação processada, como acontece na Fase 1. No entanto, o agente
ainda é capaz de mudar a direção de sua matriz moral, isso faz com que as
matrizes morais dos agentes tenham uma assinatura estatística comparável as
matrizes morais de pessoas que responderam o questionário sobre Fundamentos
Morais. O mais importante de nossa abordagem é que a complexidade vivenciada
pelo agente na Fase 1, é positivamente correlacionada com a chance da
sociedade formada por esse tipo de agente ter a mesma assinatura estatística
de pessoas liberais.

A ideia de usarmos uma descrição puramente probabilística é que está
nos fornece um arcabouço teórico coerente para lidarmos com problemas onde
existe informação incompleta\cited{CatichaA2012,Jaynes2003,Cox1946}. Além
disso, o algoritmo de aprendizado Bayesiano tem duas importantes
características, a primeira é que ele surge de forma natural no contexto
probabilístico e pode ser deduzido a partir de princípios de informação
minima\citep{CatichaA2012}. A segunda propriedade é que esse algoritmo aprende
de forma \textit{ótima}. O algorítimo de aprendizado usado em nossa modelagem
pode ser considerado ótimo pois ele pode ser obtido através da optimização
funcional do aprendizado de uma rede neural\cite{Kinouchi1992}. Além
disso, como foi mostrado por Neirotti e Caticha\cite{Neirotti2003} usando
programação evolutiva, algoritmos de classificação submetido a pressões
evolutivas por menores erros de generalização são levados a algoritmos
parecidos com o Bayesiano tanto em performance quanto em forma funcional.
Portanto, se existir uma pressão evolutiva em seres humanos para minimizar o
erros de classificação de assuntos durante interações sociais, esperamos que a
descrição Bayesiana probabilística do aprendizado seja adequada.

É marcante que alguns resultados de nossa teoria estejam em conformidade
com observações experimentais que exploram os motivos do liberalismo em
diferentes contextos. Uma importante evidência experimental, no contexto
de genética, que conecta complexidade de informação e liberalismo é
apresentada por Settle \textit{et al} \citep{Settle2010a} onde o número de
amigos que a pessoa teve na infância está correlacionada com o liberalismo,
pelo menos para pessoas que apresentam dois alelos do gene \textbf{DRD4-R7}. Os
autores desse trabalho apontam cautelosamente que seu estudo não é suficiente
para concluir que a presença de um gene especifico é causa da ideologia
política, mas sim que existem evidências de que relação entre gene e
ambiente influenciam nesse comportamento.

Ainda no contexto de \citep{Settle2010a} é possível nos perguntarmos
qual é a interpretação genética dos resultados de nossa teoria? Nossa
teoria não explica os mecanismo genéticos que fazem com que pessoas tenham
diferentes estratégias cognitivas, no entanto, podemos levantar hipóteses
sobre quais fenômenos poderiam fazer com que as pessoa tivesse acesso a
mais complexidade de informação na Fase 1, como maior números de amigos,
maior velocidade na aquisição de informação moral, prolongamento do
período da adolescência, entre outras.

O tempo de relaxação não foi usado na formulação teórica do modelo,
ele é uma consequência física do processo de troca de informação na
sociedade e portanto uma previsão do modelo. Diferentes estilos cognitivos,
através da interação social, geram diferentes tempos de adaptação. A
existência de uma transição de fase entre sociedades moralmente ordenadas
para totalmente desordenadas talvez não deva existir na realidade. No entanto,
esperamos que este modelo pode ser aplicado a outros cenários culturais
relevantes, onde grupos dos dois lados da transição podem ser encontrados.

Outra predição do modelo é que sob o aumento do parâmetro de pressão
social $\beta$ a sociedade tenderá estatisticamente a ficar mais conservadora,
como é mostrado nas figuras \ref{fig:distpa} e \ref{fig:threat}. Esse efeito
de crescimento da pressão social pode estar por atras de resultados como
os de Bonano \textit{et al} \citep{Bonanno2006} e de Nail \textit{et al}
\citep{Nail2009} sobre o crescimento do conservadorismo nos Estados Unidos
após o ataque de 11/09. No entanto, Nail \textit{et al} \citep{Nail2009a}
mostraram que não existe uma intervenção social direta em forma de ameaça
externa para que a sociedade se torne mais conservadora, sendo que um efeito
equivalente pode ser obtido pelo menos momentaneamente ao se fazer com que
o individuo se imagine numa situação de ameaça ou conforto. Isso sugere
que nossa interpretação do parâmetro $\beta$ como pressão social pode
ser entendida como um parâmetro que pode ser auto regulado dinamicamente
através da extração da informação do contexto social.  Uma possível
definição empírica e consequentemente medida de pressão social mais
adequada para o nosso modelo pode ser feita através da metodologia usada em
\citep{Gelfand2011} onde é feito um estudo de entre 33 nações para avaliar
o quão rígidas/flexível é a população em torno das normas sociais.

Uma característica importante de nossa teoria é que ela é livre de
semântica.  Ou seja, a matemática usada para definir os vetores morais
não é capaz de diferenciar se uma componente se refere a fundação moral
de justiça ou lealdade. Acreditamos que esse importante aspecto deve ser
investigado sob uma perspectiva evolutiva, de forma que se possa compreender
a emergência das diferentes dimensões morais e construir um arcabouço
matemático para a incorporação da semântica na teoria.

