\chapter*{Introdução}


A aplicação de métodos de Mecânica Estatística está se disseminando nas mais
variadas áreas do conhecimento. Esses métodos são reconhecidamente bem
sucedidos na descrição de fenômenos físicos, com eles propriedades
macroscópicas são descritas quando temos modelos do funcionamento dos
constituintes microscópicos do sistema em estudo.

O emprego desses métodos para o estudo de redes neurais também já é bem
estabelecido no meio científico, principalmente por causa de suas aplicações em
engenharia e computação. Uma outra vertente de aplicação da Mecânica
Estatística que está em contínuo crescimento, é no estudo de problemas
econômicos, e de maneira mais geral, problemas relativos ao estudo das
sociedades.

A aplicação de métodos de física em problemas de sociais foi durante muito
tempo pouco produtiva, principalmente pela dificuldade de se modelar
quantitativamente as interações entre as pessoas. No entanto, técnicas modernas
empregadas em neurociência, como imagens funcionais de ressonância magnética,
permitem avaliar a respostas do cérebro de indivíduos quando esses são
submetidos a variados tipos de interações \citep{Fowler2008}. Além disso, no
âmbito da economia e sociologia, a quantidade crescente de pesquisas envolvendo
dados numéricos, como por exemplo pesquisa de opinião, fornecem uma
oportunidade para modelarmos alguns aspectos da sociedade.

Recentemente, o psicologo social Jonathan Haidt juntamente com seus
colaboradores desenvolveram a Teoria dos Fundamentos Morais (TFM)
\citep{Haidt2007}. Essa teoria identifica um conjunto de pelo menos 5 dimensões,
valores, ou fundamentos morais que guiam o julgamento de pessoas. As dimensões
são: Justiça, Violência, Lealdade, Respeito a autoridade, Santidade ou Pureza.
O conjunto de dimensões morais no individuo particular de um indivíduo
denominado na teoria matriz moral. Usando um questionário disponível online
\citep{Quest} é possível de ponderar as componentes da matriz moral dos
respondentes através de perguntas que medem a importância (numa escala de 0 a
5) de temas ligados a cada um dos fundamentos moral.
 
Um resultado  marcante que pode ser feita observados os dados do
questionário, é que indivíduos que se declaram sua ideologia política como
liberal tendem a levar em consideração principalmente as componentes morais de
justiça e violência, enquanto indivíduos que se auto denominam conservadores
tendem a levar em consideração todos os fundamentos ou dimensões morais de forma
equivalente.

Os motivos dos diferentes tipos de ideologia política é fonte de intenso debate
entre os cientistas sociais e principalmente entre os cientistas políticos.
Essa diferenciação é comumente associada ao ambiente social, no entanto, exitem
trabalhos de neurociência que relacionam ideologia política com estratégias
cognitivas. Mais precisamente, em \citep{Amodio2007} relacionam  a resposta
cerebral ao aparecimento novidade e complexidade com ideologia politica.  Nesse
trabalho constatou-se que indivíduos que se denominam liberais tem uma maior
ativação cerebral quanto a novidade comparados com  indivíduos que se denominam
conservadores. 

Já no contexto da genética, o cientista político James Fowler juntamente com
seus colaboradores, mostram em uma série de trabalhos que pode existir um
componente genético sobre a filiação política. Mais importante para o nosso
trabalho é estudo \citep{Fowler2008}, que mostra que a ideologia liberal esta
positivamente relacionada com o numero de amigos na infância para  pessoas que
possuem dois alelos do gente de receptor de dopamina \textbf{DRD4-7R}. 

Nesse relatório, apresentarei um modelo de agentes, semelhante ao introduzido em
\citep{Caticha2011a}, que mimetiza alguns traços do aprendizado moral humano
que foram testados experimentalmente verificados.

As principais características desse modelo é que os agente tem duas fazes de
aprendizado. A Fase 1 é uma imitação do aprendizado durante o período da
infância até a adolescência, onde o modelo se caracteriza pela definição das
estratégias cognitivas dos agentes. Já na Fase 2 os agentes tem suas
estratégias cognitivas cristalizadas e o aprendizado moral se dá em uma
sociedade que discutem assuntos em torno de uma tendência. Essa tendência dos
assuntos discutidos é denominada como \textit{zeitgeist}. O quanto a sociedade
estará dispersa em torno do \textit{zeitgeist} dependerá de uma parâmetro de
pressão social. 

O texto aqui apresentado é uma versão limitada dos
principais resultados obtidos durante o meu período de doutoramento, poucos
detalhes técnicos foram incluídos. Primeira mente falarei sobre as motivações
experimentais do trabalho, em seguida introduzirei o modelo de agente e
apresentarei os principais resultados numéricos. Por fim, discutirei como os
resultados estão relacionados com os experimento. Também inclui no apêndice uma
versão simplificada sobre aprendizado Bayesiano. 
