\chapter{Ingredientes empíricos} %#{{{1
\label{chap:mft}
\begin{epigraphs}
\qitem{It is our needs that interpret the world; our drives and their
For and Against.  Every drive is a kind of lust to rule; each one has its
perspective that it would like to compel all the other drives to accept as
a norm.}
{\---- \textsc{Friedrich Nietzsche,1844-1900}}
\end{epigraphs}

Por que pessoas aparentemente bem intencionadas e de culturas relativamente
similares podem discordar de forma profunda sobre temas importantes para
sociedade?  

A missão desse capítulo é descrever para o leitor um conjunto de teorias
e experimentos que estão relacionados com o aprendizado de pessoas e que
de certa forma podem ser capturados por nossa modelagem matemática. Nós
vamos discutir sobre o que \textbf{"é"} a moral e como ela está relacionada
com a ideologia política. Discutiremos alguns fatores determinantes para
a moralidade e ideologia dos indivíduos, sendo esses fatores de natureza
diversa; variando desde de possíveis influências genéticas; diversidade
de informação moral que um indivíduo recebe e a maneira que se lida
com novas informações.

\newpage
\section{Cognição Moral} %#{{{2

A conceitualização da moralidade humana é um problema no qual
filósofos se debruçam há milênios. Em geral, os filósofos morais têm
a preocupação de definir o que é uma atitude moral, ou como os agentes
devem tomar atitudes morais enquanto o psicólogo ou o cientista moral tem
a preocupação de classificar e entender diferentes tipos de comportamento
moral \citep{Greene2003a,Casebeer2003}.

No livro \textit{The innate mind}\cite{Haidt2007a}, Jonathan Haidt e Craig
Joseph ressaltam logo no início do texto que durante a história dificilmente
os estudiosos do comportamento moral se distanciam de sua própria moral em
suas pesquisas. Um exemplo foi o embate científico ocorrido nas décadas de 70
e 80 entre entre as correntes teóricas de Lawrance Kohlberg e Carol Giligan.
O primeiro autor identificava  valores morais relacionados a \textit{justiça/
trapaça} como suficientes para definir moralidade \citep{Kohlberg1969}. A
segunda autora identificava que a moralidade também era derivada de valores
morais relacionados com \textit{cuidado / violência} \citep{Gilligan1982}.
Kohlberg e Giligan deram grandes contribuições sobre os estudos de psicologia
moral, entre elas, a exploração de dilemas éticos usando a ferramenta de
questionários em estudos longitudinais. No entanto, outros pesquisadores
ocidentais, principalmente antropólogos, identificaram que a moralidade
consiste em mais valores que justiça e cuidado. Por exemplo, para Shewder
\citep{Shweder1997a} seria suficiente para descrever a moralidade na maior
parte das culturas um conjunto de três "éticas": autonomia, comunidade
e divindade.

Dentre as muitas tradições ocidentais de filosofia moral
que influenciaram as pesquisas de psicologia moral moderna
%\footnote{Geralmente o termo Ética é denominado como a área da
%filosofia ligada ao estudo da moral ou mais concisamente a Filosofia
%Moral}
podemos destacar duas grandes correntes que despontaram durante
o período histórico do Iluminismo: \textit{consequencialista} e
\textit{deontológicas}\cited{Haidt2007a,Casebeer2003}.
As teorias morais deontológicas, com Emmanuel Kant (1724-1804) sendo o seu
mais proeminente representante, afirmam que a moral pode ser derivada a partir
do pensamento lógico ou de um pensamento racional puro, sem dar ênfase à  
consequência das ações. Já as teorias morais consequencialistas, entre
as quais a \textit{utilitarista} é a mais conhecida, afirmam que os atos
morais devem ser medidos de acordo com as consequências que eles geram
na sociedade, sendo os grandes representantes dessa corrente Jonh Stuart
Mill (1808-1873) e Jeremy Bentham (1748-1832).  Em contra partida, outra
vertente filosófica ocidental, conhecida como \textit{teoria das virtudes
morais}, que tem origem na Grécia antiga, da qual filósofos como Platão e
Aristóteles são grandes representantes, afirma que o comportamento moral
deve ter o intuito de cultivar virtudes e evitar vícios.

De acordo com William D. Casebeer\citep{Casebeer2003}, seriam necessários
diferentes esforços cognitivos, que podem ser medidos experimentalmente,
para executar atitudes morais em cada uma das desses três filosofias. O uso
das éticas formalistas e consequencialistas demandariam que as decisões
morais fossem feitas por áreas de cognição superiores como córtex pré-frontal
e algumas partes destinadas a cognição sensorial. Já a ética da virtude
necessitaria da coordenação dessas partes com outras associadas com o
processamento de emoções. Em geral, a inferência sobre a ativação cerebral e
decisão moral é feita através de estudos de \textbf{fMRI} onde a atividade
cerebral do indivíduo é monitorada enquanto ele deve se concentrar na resolução
de dilemas éticos. Para uma revisão mais recente sobre essa metodologia
indicamos a referência \cited{Christensen2012}. 

Um crescente corpo de experimentos \citep{Greene2004,Christensen2012}
evidenciam que a cognição moral necessita da coordenação tanto de regiões
cerebrais relacionadas ao processamento de emoção quanto de regiões de
cognição mais avançadas que estão ligadas ao planejamento de ações
e raciocínio lógico, corroborando uma visão mais próxima da éticas da
virtudes sobre a cognição moral \citep{Casebeer2003}.

Da mesma maneira que foi considerado em\citep{Caticha2011a}, mais importante
para o nosso trabalho é o fato de que violações morais causam uma
grande reação negativa associadas a ativação de áreas
relacionadas a cognição emocional e social. Isso evidencia
que uma parte importante do processamento moral é de origem intuitiva e
automática\cite{Haidt2001,Greene2002,}, sendo que as regras de conduta
moral mais intricadas seguem de uma racionalização da conduta moral que
é feita a posteriori. Para uma discussão detalhada sobre os componentes
emocionais da intuição moral e seus substratos cerebrais sugerimos ao
leitor as referências \citep{Greene2001,Moll2005,Woodward2008}.

Não estamos afirmando, no entanto, que emoções ou intuições são os
únicos componentes por trás da cognição moral, pois como é sugerido
experimentalmente, respostas negativas automáticas podem ser sobrepujadas
por respostas mais utilitaristas, que são feitas recrutando-se áreas
do córtex pré-frontal, e ocorrem principalmente quando pessoas devem
julgar difíceis dilemas éticos pessoais com importantes consequências
sociais\cite{Pizarro2003,Koenigs2007}.  De maneira análoga, um experimento
conduzido recentemente \cite{Rand2012} mostra que tomar decisões rapidamente
privilegia cooperação, enquanto o contrario, faz com que o indivíduos
tomem  decisões "racionais" que favorecem a si mesmos.

Finalmente, consideraremos uma perspectiva mais descritiva do comportamento
moral e que é bem representada pela definição de sistema moral dada por
Jonathan Haidt\cited{Haidt2012,Haidt2010}.
\begin{description}
    \item
        \textbf{O sistema moral} é um conjunto de valores, virtudes, normas,
        práticas, identidades, instituições, tecnologias e mecanismos
        psicológicos evoluídos que trabalham para agregar, suprimir ou regular
        o auto interesse e fazendo que sociedades operantes sejam viáveis
        \footnote{ tradução livre de:\textit{ Moral System are interlocking
        sets of values, virtues, norms, practices, identities, institutions,
        technologies and evolved psychological mechanisms that work to gather
        to suppress or regulate self-interest and make operative societies
        possible.} }.
\end{description}


\subsection{Teoria dos Fundamentos Morais}%#{{{3

A Teoria dos Fundamentos Morais foi inicialmente proposta pelos psicólogos
sociais Jonathan Haidt e Craig Josef em \cited{Haidt2004}. Nesse trabalho os
autores procuraram na literatura fundamentos ou dimensões morais que poderiam
ser inatos a todos os seres humanos. Usando uma abordagem metaempírica os
autores contabilizaram a frequência com que palavrascom conteúdo moral
\footnote{A lista de palavras com conteúdo moral selecionadas pelos autores
pode ser encontrada na referência\citep{Graham2009}} apareceram em cinco
trabalhos acerca do comportamento humano. Dois trabalhos que se propunham
descrever o que é universal nos humanos \citep{Brown1991,Fiske1991}, dois que
analisavam o que é culturalmente variável\citep{Schwartz1990,Shweder1997a} e
um trabalho sobre teorias evolutivas da "moral" encontrada em primatas  além
do \textit{Homo Sapiens}\citep{Waal1997}.  As palavras selecionadas foram
agrupadas em cinco dimensões ou fundamentos morais que são justificáveis
como produtos de diferentes desafios evolutivos
\footnote{
    Os nomes dimensões morais são tradução livre respectivamente de:\\
    \textit{Fairness / Cheating};\\ \textit{Care / Harm};\\
    \textit{(in-group) Loyalty / Betrayal}; \\
    \textit{Authority / Subversion};\\ \textit{Sanctity (or Purity)/ Degradation}.}.

\begin{description}
\item \textbf{Justiça / Trapaça}: Surge como produto do desafio evolutivo de
aproveitar recompensas devido a cooperação sem que se seja explorado. Torna as
pessoas sensíveis a indicativos que outra pessoa possa ser um bom (ou mal)
parceiro de colaboração e altruísmo recíproco.

\item \textbf{Cuidado / Violência}: Surge como produto do desafio evolutivo 
de cuidar de crianças vulneráveis. Isso torna as pessoas sensíveis a sinais 
de sofrimento e necessidade, fazendo com que desprezem crueldade e queiram 
cuidar de quem está sofrendo. 

\item \textbf{Lealdade (a grupos)/ Traição}: Surge como produto do desafio
evolutivo de se manter coalizões. Torna as pessoas sensíveis a sinais de que o
outro é ou não um bom membro para times. Faz com que queiram recompensar esse
tipo de indivíduo e faz com que as pessoas queiram machucar, ostracizar, ou
até mesmo matar aqueles que consideram traidores.

\item \textbf{Respeito à autoridade / Subversão}: Surge como resposta aos
desafios evolutivos que forjam relações sociais hierárquicas. Torna as pessoas
sensíveis a sinais de ranque ou status social, e a sinais de que pessoas estão
agindo de maneira adequada ou não de acordo com suas posições.

\item \textbf{Santidade (ou Pureza)/ Degradação}: Surge como resposta ao desafio
evolutivo do dilema do onívoro\footnote{ Experimentar um novo tipo de
alimento que pode ser mais nutritivo (ou não) e correr o risco de ser alvo
de predação ou de intoxicação alimentar, ou permanecer na mesma dieta},
e também surge como resposta ao desafio mais geral de viver sob o risco
de contrair doenças devido a parasitas. Essa dimensão leva em conta o
comportamento de sistema imunológico fazendo com que os indivíduos se
preocupem com diversos objetos simbólicos e ameaças. Faz com que pessoas
possam investir valores extremos a objetos de forma irracional.
\end{description}

\begin{figure} 
\centering
\includegraphics[scale=0.6]{Figures/haidt_science}
\caption{Figura retirada de \citep{Haidt2007}: Fundamentos morais de
conservadores e de liberais. No eixo das coordenadas é apresentado a relevância
do fundamento morais um julgamentos. No eixo das abíssicas é apresentado a
filiação política dos respondentes. Percebe-se que liberais julgam como mais
importantes os fundamentos de Violência e Justiça, enquanto conservadores
julgam com mesma importância os outros fundamentos morais, lealdade a grupo,
respeito a autoridade e pureza. A gráfico é obtido a partir das repostas de
cidadães americanos do questionário hospedado em \citep{Quest}. Cidadães de
outras nacionalidades apresentam resultados semelhantes.}
%Liberal versus conservative moral foundations. Responses to 15 questions about
%which considerations are relevant to deciding “whether something is right or
%wrong.” Those who described themselves as “very liberal” gave the highest
%relevance ratings to questions related to the Harm/Care and
%Fairness/Reciprocity foundations and gave the lowest ratings to questions about
%the Ingroup/Loyalty, Authority/Respect, and Purity/Sanctity foundations. The
%more conservative the participant, the more the first two foundations decrease
%in relevance and the last three increase [n = 2811; data aggregated from two
%web surveys, partially reported in (41)]. All respondents were citizens of the
%United States. Data for 476 citizens of the United Kingdom show a similar
%pattern. The survey can be taken at www.yourmorals.org.}
\label{fig:haidt}
\end{figure}

\newpage
Um dos pontos mais importantes da Teoria dos Fundamentos Morais é a
constatação experimental de que em média as pessoas utilizam diferentes
conjuntos de fundamentos morais dependendo de sua ideologia política
\citep{Haidt2007,Haidt2009}\cite{Haidt2007,Haidt2009}.  Como observa-se
na figura \ref{fig:haidt}, pessoas que denominam sua filiação política
como {\blue liberal} tendem a fazer julgamentos morais levando em conta
principalmente as dimensões {\blue justiça e cuidado}, no entanto, {\red
conservadores} também levam em consideração com a mesma importância as
outras três dimensões de moralidade, {\red lealdade, respeito à autoridade,
santidade}.

Usando um questionário, que está disponível online\cited{Quest}, é
possível estimar (numa escala de 1 a 6) a importância que um indivíduo dá
a cada um dos fundamentos morais. Ao responder o questionário o indivíduo
primeiramente declara sua afiliação política numa escala de 1 a 7, sendo
1 muito liberal e 7 muito conservador.  Na figura \ref{fig:moralmatrix}
vemos um exemplo de matriz moral de um indivíduo (verde) comparada com a
matriz moral média de indivíduos conservadores (vermelho) e liberais (azul).
O conjunto de fundamentos morais de um indivíduo é chamado pelos proponentes
da Teoria dos Fundamentos Morais de \textbf{matriz moral} \footnote{O termo
matriz moral é  usado no sentido de fundação ou alicerce moral}.

\begin{figure} 
    \centering
    \includegraphics[scale=0.4]{Figures/moralmatrix}
    \caption{
        Exemplo de medida da matriz moral de um indivíduo (verde), comparada
        com a matriz moral media de indivíduos liberais (azul) e conservadores
        (vermelho). A ordem de apresentação dos fundamentos da esquerda
        para direita é violência, justiça, lealdade, autoridade e pureza.
    }
    \label{fig:moralmatrix}
\end{figure}

Tendo em vista que grande parte dos filósofos e cientistas sociais se
identificam com ideologias liberais fica claro o motivo pelo qual eles
definem moralidade como um conjunto de regras para o convívio social que
está baseado nas dimensões justiça e cuidado.

Para tentarmos responder por que pessoas com ideologias políticas distintas
usam de forma diferente as intuições morais, iremos discutir nas próximas
sessões os mecanismos de aprendizado das pessoas e os estudos que investigam
porque as pessoas optam por diferentes ideologias políticas.

\subsection{O que está faltando ?} %#{{{3

Uma pergunta evidente é se existem outras dimensões morais relevantes. A
resposta para isso é provavelmente sim. Recentemente, Jonathan Haidt e
seus colaboradores incluíram a dimensão moral   
\begin{description}
    \item \textbf{liberdade/opressão}: que surge como resposta ao desafio
        evolutivo de se viver em grupos com indivíduos, que se tiverem a chance
        irão dominar, intimidar e constranger os outros\citep{Haidt2012}. 
\end{description}

Ao incorporar essa nova dimensão, de acordo com os proponentes da Teoria
dos Fundamento Morais, é possível entender com mais profundidade outras
ideologias políticas que podem estar de fora do espectro liberal/conservador.
Um exemplo é a ideologia libertária, que tem liberdade/opressão como a
principal dimensão moral \cite{Iyer2012,Haidt2012}.

Em nosso trabalho, levamos em consideração somente as cinco primeiras
dimensões morais e o espectro político liberal/conservador para que possamos
fazer a comparação entre os resultados de nossa modelagem com os dados
experimentais que temos em mão.

\section{Mecanismos neurológicos do \\aprendizado por reforço} %#{{{2
\label{sec:ApRe}

É inegável que os seres humanos aprendem através das consequências de
seus atos \citep{Holroyd2002}.  O aprendizado por reforço pode ser pensado
como o processo de aprendizado por tentativa e erro\citep{Shah2012}. Ou seja,
esse tipo de aprendizado é caracterizado  pelo senso comum de que se uma
ação gera ou é seguida de uma sensação satisfatória ela tem uma maior
probabilidade de ocorrer novamente, caso contrário, se a ação é seguida
de um resultado negativo a probabilidade é menor\cited{Holroyd2002,Barto1995}.

De fato, erros são importantes fontes de informação e regulação de
processos cognitivos, no entanto, os mecanismos pelos quais as pessoas
detectam e corrigem os seus erros não são totalmente claros, sendo alvos 
de estudos de grande interesse da comunidade científica\cite{Yeung2004}.
Em particular, estudos de potenciais cerebrais relacionados a eventos
\footnote{ tradução livre de: \textit{Event-related brain potential} }
têm revelado as respostas neurais que ocorrem após erros. Esses sinais são
denominados pela literatura como negatividade relacionadas ao erro \footnote{
tradução livre de: \textit{error-related negativity} }. A área cerebral que
tem a maior probabilidade de gerar esse sinal é o córtex cingulado anterior,
que também é uma área associada à monitoração de competição ou conflito
\citep{Yeung2004}. A ativação dessa área também é confirmada através
de imagens de \textbf{fMRI} \citep{Yeung2004}. Uma descrição mais detalhada
desse tipo de experimentos será feita na seção \ref{subsec:neuropol}

Além disso, existe uma forte relação do aprendizado por reforço e a
ativação neurônios dopaminérgicos localizados nos núcleos basais (ou
gânglios basais)\citep{Holroyd2002,Redgrave2006b}. De fato, a dopamina é
central em processos relacionados a recompensas, apesar do seu exato papel
ainda ser controverso. Uma das hipóteses mais aceitas para o seu papel é
que durante o aprendizado ela é usada como um antecipador de recompensas
\citep{Flagel2011}.

Alguns experimentos \citep{Klucharev2009,Campbell-Meiklejohn2010} mostram
que o córtex anterior cingulado é ativado quando indivíduos têm
opiniões conflitantes com a norma imposta por um grupo.  Isto indica
que os mecanismos neuronais de aprendizado por reforço corroboram a
teoria de influência social moderna que será discutida com maiores
detalhes na sessão \ref{subsec:insoci}.  É importante salientar que
o córtex anterior cingulado também é ativado durante a dor física
\citep{Somerville2006,Eisenberger2003}. Estudos recentes mostram que a dor
física compartilha uma grande parte das representações somatossensorias
com a percepção de forte exclusão social\citep{Kross2011}. Com isso,
é possível concluir que a exclusão social gera uma sensação correlata
à dor física.

\section{Ideologia Política} %#{{{2

Os motivos dos diferentes tipos de ideologia política é fonte de intenso debate
entre os cientistas sociais e principalmente entre os cientistas políticos.
Além disso, mais do que explicar a diversidade de ideologias políticas, a
explicação do conservadorismo soa como um enigma na aérea de ciências
políticas. Em 2003, um influente trabalho de John Jost e colaboradores
\citep{Jost2003} introduziu a teoria na qual o conservadorismo político é
\textit{motivado por cognição social}. 

O termo \textit{motivado por cognição social} foi introduzido para fazer
referência a que o conservadorismo, como qualquer sistema de crenças,
deve ser derivado de algumas necessidades psicológicas, que variam de acordo
com o ambiente (cultural, social, genético) imposto ao indivíduo.  No caso
do conservadorismo, as necessidades psicológicas devem estar ligadas com a
manutenção do \textit{status quo}.  Neste trabalho, os autores apresentam
uma metanálise de 12 pesquisas com um total 22 mil com participantes de
12 países. Eles propuseram uma teoria de psicologia abrangente incorporando
várias outras teorias sobre os motivos do conservadorismo. Entre os motivos
das teorias selecionadas estão motivos de personalidade (autoritárias,
dogmáticas, intolerantes a ambiguidade), necessidade epistêmicas e
existências (por fechamento, por controle de foco, por controle de medo) e
racionalização ideológica (dominância social e sistemas de justificativa).

Em nosso trabalho estamos interessados em algumas características
comportamentais que se diferenciam com a ideologia política e que estão
relacionadas a diferentes estratégias cognitivas para se lidar com conflitos
e novas informações. Por exemplo, um resultado bem conhecido nas ciências
sociais relaciona ideologia política com diferentes componentes dos 5 grandes
traços de personalidade\footnote{Os 5 grandes traços de personalidades
são: Neuroticismo, extroversão, sociabilidade, escrupulosidade, abertura
para experiências. Eles foram introduzidos na literatura em 1961 por
Tupes e Christal e pode ser encontrado em \citep{Tupes1961}}.  Enquanto
pessoas liberais apresentam traços maiores de abertura à experiência,
conservadores estão positivamente correlacionados com traços de
escrupulosidade\cite{Gerber2010}.

Em um trabalho interessante feito em 2009 \cite{Shook2009} Shook e
Fazio examinaram a diferença do perfil exploratório entre liberais e
conservadores em uma tarefa de aprendizado probabilístico. Nesse experimento,
foi apresentado para os participantes, em uma tela de computador, um conjunto
de imagens de sementes que variavam de aparência e tamanho. O objetivo do
participante era descobrir quais os formatos de grãos são vendidos com maior
lucro. Verificou-se que pessoas liberais aderiam à estratégias
mais arriscadas e exploratórias, mas que conferiam uma vantagem no fim do
experimento, já os conservadores aderiam a estratégias mais prudentes e menos
informativas mas que lhes garantiam uma vantagem no início do experimento.
Como veremos em seguida, essas características foram observadas em  estudos
feitos tanto no âmbito neurocientífico quanto no genético.

\subsection{Ideologia Política, Estilos Cognitivos \\e Neurociência} %#{{{3
\label{subsec:neuropol}

A aplicação de ferramentas usadas em neurociência começaram a ser usadas
recentemente nos estudos sobre comportamento político \cited{Jost2011}. Um
exemplo importante para o nosso modelo é o experimento de monitoração de
conflitos e resposta à novidade feito por Amodio e colaboradores em 2007
\cited{Amodio2007} com o intuito de medir a diferença das estratégias
cognitivas entre liberais e conservadores. Para tanto, foi executado o
experimento conhecido como \textit{Go/NoGo}. Nesse tipo de experimento
os participantes devem cumprir uma tarefa quando são submetidos a um
estímulo \textit{Go} frequente e não cumprir a tarefa quando submetido a
um estímulo \textit{No Go} pouco frequente \footnote{Exemplos de estímulos
\textit{Go/NoGo} podem ser a visualização de símbolos geométricos simples
como quadrados, triângulos, etc, projetadas em um monitor, e um exemplo de
tarefa é pressionar um botão de mouse usando um dedo}.

O estímulo \textit{Go} é repetido de forma que o participante se acostume
a ele, isso que faz com que o sinal \textit{NoGo}, que é menos frequente,
cause uma sensação de surpresa no participante, além de fazer com que ele
tenha de controlar o impulso de executar a ação.  O termo monitoração
de conflito designa o mecanismo de detecção de  quando uma resposta a
um estímulo habitual entra em conflito um estímulo imediato. Como foi
discutido na sessão \ref{sec:ApRe}, a resposta cerebral a esses estímulos
são comumente relacionadas ao córtex anterior cingulado \citep{Yeung2004}.

Na figura \ref{fig:amodio}(a) é mostrado a relação entre filiação
política e o índice de monitoração de conflito. Esse índice consiste
basicamente da diferença entre as amplitude máximas dos potenciais de
respostas dos estímulo \textit{Go} e \textit{NoGo}. Com isso, vemos que o
liberalismo político está positivamente correlacionado com a diferença
de respostas entre os estímulos. Na figura \ref{fig:amodio}(b) estão
exemplos de curvas com os a diferença dos potenciais de resposta entre os
estímulo \textit{Go} e \textit{NoGo} em função do tempo para conservadores
e liberais. Já na figura \ref{fig:amodio}(c) é apresentado a localização
da fonte do sinal do potencial de resposta como sendo o córtex cingulado
anterior. Esses resultados experimentais foram replicados de forma independente
em \citep{Weissflog2010}.

\begin{figure}
    \includegraphics[width=\linewidth]{Figures/amodio} 
    \caption{
        Figura retirada de \citep{Amodio2007}. Relação entre filiação
        política e índice de monitoração de conflito.  (a) diferenças
        de amplitude dos sinais em função da orientação política,
        vemos que o liberalismo político está positivamente correlacionado
        com a diferença das respostas entre os estímulos \textit{Go}
        e \textit{NoGo}. (b) Exemplos de curvas com os a diferença entre
        potenciais de resposta entre os  estímulo \textit{Go/NoGo} em função
        do tempo para conservadores e liberais. (c) Localização da fonte
        do potencial de resposta como sendo o córtex cingulado anterior.
    } 
    \label{fig:amodio}
\end{figure}

Com isso, tem-se que indivíduos liberais tem uma maior ativação cerebral
com a ocorrência de eventos inesperados quando comparados com a ativação
cerebral de eventos que são previsíveis. De maneira análoga, percebemos que
conservadores tem uma maior ativação cerebral para eventos previsíveis em
relação aos imprevistos quando comparado com liberais.  De acordo com Jost
e Amódio\citep{Jost2011}, esse resultado juntamente com outras evidências
científicas, sugere que o liberalismo é associado com uma forte motivação
de procura de novas informações e integração de informações conflitantes
para que o indivíduo consiga entender a realidade.

\subsection{Ideologia política, genética e dopamina:\\Primeiras evidências} %{{{3

Ainda é comum nas pesquisas dos cientistas sociais o paradigma de que a
ideologia política é decorrente somente do contexto social. No entanto,
está crescendo o número de trabalhos que relacionam a filiação
política de indivíduos não somente com o meio social no qual ele
está inserido mas também com outras componentes de origem biológica
\citep{Fowler2008,Dawes2009,Settle2010,Hatemi2011}.

Em $1984$, em um estudo\cite{Martin1986} pioneiro, o geneticista Nicolas
Martin juntamente com seus colaboradores sugeriu que genes podem exercer
pressões sobre as atitudes de indivíduos em relação a tópicos como
aborto, imigração, pena de morte, pacifismo entre outros. Eles usaram nesse
trabalho a técnica clássica de estudo com gêmeos para inferência genético
/ comportamental: comparando gêmeos monozigóticos com heterozigóticos.
Em média, gêmeos monozigóticos compartilhavam crenças politicas mais
do que os heterozigóticos. Haja visto que com grande probabilidade gêmeos
crescem sobre um mesmo contexto familiar, os autores chegaram a conclusão
que fatores genéticos têm um papel significante sobre a atitude política
dos indivíduos\cite{Buchen2012}. Apesar de ser um assunto polêmico
dentro das ciências políticas, as implicações desse trabalho passaram
desapercebidas durante pelo menos 20 anos. Em 2005  os cientistas políticos
Hibbing e Alford reanalisaram os dados usados por Martin incorporados a
outros conjuntos de dados e constataram novamente uma grande correlação
entre genética e visão política\cite{Alford2005}.  A partir daí, uma
série de outros trabalhos começaram a incorporar informações genéticas
a pesquisa de ciências políticas.  Recomendamos ao leitor a discussão
feita por Smith e colaboradores \cite{Smith2011} que ilustra a trajetória
entre genética e atitude política incluindo 4 níveis intermediários:
biológico, cognitivo/processamento de informação, personalidades/valores
e ideologia levando em conta a influência de ambiente entre eles.

Mais importante para o nosso trabalho é o estudo desenvolvido pelo
cientista político James Fowler e seus colaboradores no qual, usando dados
do \textit{National Longitudinal Study of Adolescent Health}, conseguiram
uma associação entre a filiação politica e o número de amigos na
adolescência para pessoas que possuíam duas cópias do alelo \textbf{7R} do
gene de receptor de dopamina \textbf{D4} (\textbf{DRD4-7R})\cited{Dawes2009}.

A dopamina é um neurotransmissor da família das catecolaminas que possui
diferentes funções no cérebro que são relacionadas as funções de
seus 5 tipos de receptores,(\textbf{D1,D2,..,D5}). O receptor de dopamina
\textbf{DRD4} é uma proteína sintetizada por um gene que leva o mesmo nome,
esse gene existe em pelo menos 3 formas polimórficas diferentes entre elas
a forma que possui o alelo \textbf{7R} \citep{Dawes2009, Tol1992}. Existem
diversos estudos que relacionam a presença desse gene com características
comportamentais de busca de novidade, impulsividade, extravagância,
tendência exploratória, enquanto ausência desse gene está relacionada com
características como rigidez de pensamento, lealdade. Um estudo recente,
indica que diferença em perfis de exploração de indivíduos  podem ser
previstos em média a partir de genes dopaminérgicos \citep{Frank2009}.
%colocar mais referencias aqui

O programa \textit{National Longitudinal Study of Adolescent Health} é um
estudo longitudinal feito nos Estado Unidos com a intensão de coletar dados
representativos da população. O estudo é feito através de questionários
respondidos tanto individualmente como através de entrevistas feitas nas
casas do participantes. O objetivo do questionário é a coleta de dados
relativos aos aspectos econômicos, sociais, familiares e educacionais,
e relacioná-los com os estados de saúde e bem estar desses indivíduos.

\begin{figure}
    \centering
    \includegraphics[scale=0.4]{Figures/DRD4_Fowler}
    \caption{
        Figura retirada de \citep{Dawes2009}: A esquerda esta o ajuste
        linear da tendência entre ideologia política e número de amigos
        para pessoas com duplos alelo do $R7$. A direita aponta que os
        autores não encontraram nenhuma relação entre números de amigos
        e filiação política para quem não apresentava alelos longos $R7$.
    }
    \label{fig:DRD4}
\end{figure}

Usando os dados do \textit{National Longitudinal Study of Adolescent Health}
eles obtiveram os testes com marcadores genéticos de 2.574 indivíduos entre os
quais estava o gene \textbf{DRD4}. Desse total, $33\%$ apresentavam uma cópia
do alelo \textbf{7R}, $5\%$ duas cópias, $62\%$ das pessoas não apresentavam
cópias do alelo.  A principal contribuição desse trabalho é apresentada
no gráfico \ref{fig:DRD4} onde é mostrado a correlação entre o número de
amigos que o indivíduo tinha na sua adolescência e sua filiação política
para pessoas com dois alelos \textbf{DRD4-7R}. Mais especificamente,
como mostrado na figura esquerda de \ref{fig:DRD4} quanto mais amigos
esses indivíduos tiveram em sua adolescência maior a probabilidade desses
indivíduos se tornarem liberais no início de sua vida adulta. No entanto,
para indivíduos que não apresentavam nenhuma cópia do alelo, não foi
encontrada nenhuma correlação entre o número de amigos (numa escala
de 0 a 10) e a filiação política (numa escala de 1 a 5 sendo 1 muito
conservador e 5 muito liberal). Já para pessoas que apresentavam somente
um alelo, a correlação entre filiação politica e número de amigos não
foi estatisticamente significante quando comparado a outras quantidades
pesquisadas.
%O número de amigos foi obtido através da pergunta no questionário

Como é salientado pelos autores, o resultado obtido nesse trabalho não
é definitivo sobre o  ponto de vista de uma possível relação causal
entre gene e ideologia política, no entanto, ele mostra algumas pistas
sobre possíveis marcadores genéticos e relações sociais que podem ser
importantes para influenciar a filiação politica do indivíduo.

Concluímos assim, que devido às evidências de genéticas, de traços
de personalidades, de relações sociais na adolescência, entre outras,
que o liberalismo político deve estar positivamente correlacionado com a
quantidade ou diversidade de informações morais indivíduos foram expostos
no período de maior formação cognitiva.

\section{Pressão Social} %#{{{2
\label{sec:pressao}

Um dos grandes desafios das ciências sociais é medir a influência que os 
indivíduos têm entre si e que a sociedade tem sobre os indivíduos. Nesta seção
falaremos um pouco sobre a teoria moderna de influência social e também
discutiremos como ameaças a grupos alteram a ideologia política de indivíduos.

\subsection{Influência do grupo} %#{{{3
\label{subsec:insoci}

De acordo com \cited{Abrams1990} é possível definir três tipos de influência
social: normativa, informacional e referente informacional (tradução livre de:
\textit{normative, informational, referent informational}). As duas primeiras
(normativa, informacional) são influências interpessoais e a última
(influência referente informacional) está relacionada à noção de pertencer
a um grupo.  A influência normativa é caracterizada quando um indivíduo
expressa opiniões ou age publicamente de acordo com um padrão  para evitar
punição social ou obter alguma recompensa social.  A influência informacional é
o mecanismo de influência comum quando membros de um grupo experienciam
incerteza subjetiva e a falta de evidências objetivas para se avaliar um
estímulo. A influência social surge pois, a fim de diminuir a incerteza, o
indivíduo engaja em comparações sociais com os outros membros do grupo. 
Já a influência referente informacional contabiliza o quanto a percepção de
pertencimento a um grupo muda a opinião do indivíduo. 
            
Os experimentos clássicos de Sherif e Asch \cited{Sherif1937,Asch1956}
exemplificam a influência que grupos têm sobre
indivíduos\citep{Caticha2011a}. No experimento de Sherif, os participantes
são colocados em uma sala escura e têm a tarefa de julgar se um ponto de
luz projetado na parede está se movimentando ou não, sem que eles saibam
que de fato a projeção é estática.  Esse tarefa é repetida diversas
vezes com os participantes sozinho na sala ou em grupo. Quando em grupo,
as estimativas dos participantes convergem para um padrão especifico, ou
seja, os membros do grupo acabam entrando no consenso de que a projeção
está parada ou se movimentando.

\begin{marginfigure}
\linespread{1}
\centering
\includegraphics[width=\textwidth]{Figures/Asch}
\caption{
Par de cartas apresentadas nos experimento do Asch. À esquerda a linha de
referência e à direita as três linhas para comparação. Figura retirada de
\citep{AschWiki}.}
\label{fig:Asch}
\end{marginfigure}

No experimento de Asch, os participantes têm de comparar tamanhos de listras
verticais apresentadas em duas cartas. Na primeira carta existe 1 linha vertical
como um comprimento de referência. Na segunda carta existem 3 linhas verticais
com tamanhos distintos onde uma das linhas tem o mesmo comprimento da
linha de referência, como pode ser visto na figura \ref{fig:Asch}. 
O participante tem de julgar qual das linhas da segunda carta tem o mesmo
tamanho da linha de referência da primeira carta. 

O participante é colocado numa sala juntamente com um grupo de
\textit{confederados}, que sabem o verdadeiro objetivo do experimento.
Ele deve expressar sua opinião sobre a tarefa somente depois de ouvir a
opinião de todos os confederados. A influência do grupo é verificada
pois os confederados escolhem entre as opções de linhas verticais uma
alternativa que é claramente errada. Isso faz com que a maioria dos
participantes escolham a mesma opção do grupo.

O experimento de Asch é usualmente interpretado como um exemplo de
predominância da influência normativa, já que existe uma
norma social clara e os participantes devem expressar uma opinião contrária
à que teriam caso não tivessem de expressá-la publicamente. No entanto,
essa interpretação não está fora de questionamento, pois, como argumenta
\citep{Abrams1990}, pode-se pensar que o objetivo do experimento não é
tão claro para o participante; sendo assim, a influência predominante é
do tipo informacional e ocorre através de complacência, ou seja, apesar
de não concordar com o grupo, o participante se submete a ele com o intuito
de diminuir sua incerteza sobre o estímulo.

Já o experimento de Sherif é um exemplo clássico no qual a influência
informacional age de forma preponderante pois o grupo cria uma norma própria
a partir da análise de um estímulo ambíguo, e, além disso, o estímulo
é realizado numa sala totalmente escura, fazendo com que os indivíduos
participem do experimento de forma anônima e discreta para os outros membros
do grupo.

Como estudado experimentalmente em \citep{Abrams1990}, os dois tipos de
influências interpessoais (normativa e informacional) dependem de que
os indivíduos se percebam como membros do grupo. Para isso os autores do
estudo refizeram os experimentos de Sherif e Asch com a diferença de que
a noção de pertencimento de grupo dos participantes era salientada. Com
isso, eles sugerem que os tipos de influência social interpessoal são casos
particulares de influência referente informativa quando o indivíduo tem
completa noção de pertencimento ao grupo.  Uma revisão mais atual sobre
os processos de influência social pode ser encontrada em \citep{Cialdini2004}.


\subsection{Ameaças e conservadorismo} %#{{{3

De acordo com \cited{Giannakakis2011}, a Teoria de controle de terror \footnote{
tradução livre de:\textit{Terror management theory} } \citep{Greenberg1992}
providencia umas das mais proeminentes explicações para intolerância
e viés contra diferenças pessoais. Por essa teoria, o instinto de auto
preservação aliado com a consciência da mortalidade cria no indivíduo um
potencial terror paralisante \citep{Arndt1997,Solomon2004}.  Para controlar
esse terror, pessoas têm a necessidade de ter fé em visões de mundo
estáveis, e se sentirem membros com valor dentro de um universo que têm algum
sentido. Essa visão de mundo é culturalmente defendida através de crenças
sobre a realidade. Com isso, as pessoas são fortemente motivadas em manter
a fé em suas visões de mundo e agirem de acordo.

Dentro da teoria de manutenção de terror a hipótese da saliência da mortalidade
diz que ao relembrar um indivíduo de sua mortalidade ele tende a aumentar sua
necessidade pelas estruturas de sua visão de mundo. Diversos estudos
corroboram essa hipótese demostrando que estímulos relacionados com morte
resultam na defesa da visão de mundo principalmente através de reações
negativas em relação a pessoas diferentes e reações positivas em relação a
pessoas similares \citep{Rosenblatt1989,Burke2010}. 
Portanto, uma possível implicação da teoria de controle de terror no âmbito
de ideologia política é que ameaças fariam com que liberais se tornassem mais
liberais e conservadores mais conservadores\cited{Nail2009}. 

No entanto, o modelo de cognição social motivada \citep{Jost2003} prevê
que ameaças fazem com que tanto liberais quanto conservadores tenham uma
tendência ao conservadorismo. De fato, também existe uma extensa literatura
\footnote{Consultar \citep{Jost2006,Nail2009} e suas respectivas referências.}
que corrobora com a ideia de que situações de ameaças aumentam as tendências
conservadoras dos indivíduos independentemente de sua ideologia política.

\begin{figure}
    \centering
    \includegraphics[scale=0.7]{Figures/nail_threat}
    \caption{
        Figura retirada de \citep{Nail2009}. Índice de atitude
        política média medida antes e depois dos ataque terrorista de
        11/09/2001. Quanto maior o índice de atitude política maior a
        ideologia conservadora.
    }
    \label{fig:nail}
\end{figure}

Por exemplo, alguns estudos apontam que ameaças à sociedade como os
atos terroristas de 11 de setembro tornam as pessoas mais
conservadoras\citep{Bonanno2006,Nail2009}. Na figura \ref{fig:nail}
apresentamos o resultado principal do artigo \citep{Nail2009}. Neste trabalho,
através de um questionário foi medido um índice de atitude politicas antes e
depois do atentado de 11/09/2001 para diferentes grupos de pessoas que
auto denominaram sua ideologia política. 

O efeito de crescimento do conservadorismo também ser pode obtido em
condições de laboratório em experimentos onde os participantes são
induzidos a se sentirem ameaçados, por exemplo, fazendo os participantes
pensarem sobre situações de injustiça ou salientando suas mortalidades
\cited{Nail2009a}.

