\chapter{Introdução}

\begin{epigraphs}
\qitem{
Let us apply to the political and moral sciences the method founded on
observation and calculation, the method with has served us so well in the
natural sciences.
}
{\---- \textsc{Pierre-Simon Laplace, 1749-1827}}
\end{epigraphs}

A Física atual atingiu um grau de sofisticação e controle
impressionante. Com arcabouço matemático relativamente consistente, nós
somos capazes de entender ou prever com grande precisão o comportamento de
eventos e objetos de escalas subatômicas até escalas cosmológicas. Isso se
torna ainda mais impressionante quando comparamos esse poder de descrição
do \textit{universo físico} com a nossa baixa capacidade de predição do
comportamento humano.

Além da grande variabilidade e complexidade dos fenômenos sociais,
um dos motivos da dificuldade de compreensão dos mesmos são os nossos
vieses cognitivos. Entre eles, a tendencia de achar que fenômenos
sociais são simples de entender. Por exemplo: no livro \textit{Tudo
é Óbvio}\cite{Watts2011} o sociólogo Duncan Watts informa ao leitor que
durante a segunda guerra mundial soldados de origem rural apresentavam melhores
índices de bem-estar quando comparados com os de origem urbana. De fato,
essa informação parece bem plausível já que é fácil de imaginar que
na terceira década do século passado a vida em áreas rurais era muito
mais complicada do que em áreas urbanas o que possibilitaria uma melhor
adaptação dessas pessoas durante a guerra. No entanto, o grupo de pessoas
que realmente reportaram melhores índices de bem-estar foram os de origem
urbana; note que essa informação também é bastante plausível já que
é fácil de imaginar que pessoas de origem urbana estão mais adaptadas em
viver em espaços pequenos e com alta concentração e também estão mais
acostumadas a lidar com vários níveis de hierarquia. Com esse argumento,
vemos que, de fato, os fenômenos sociais devem ser estudados cuidadosamente,
e principalmente levando em conta dados experimentais, já que argumentos
de bom senso nem sempre revelam a natureza do fenômeno de forma acurada.

Até o fim do último século os fenômenos sociais recebiam pouca atenção
da comunidade de Física; no entanto, com o grande poder de processamento
dos computadores modernos, diversos modelos simples de dinâmica
coletiva foram propostos a partir de princípios de senso comum, para
modelar alguns fenômenos sociais. De fato, a aplicação de métodos de
Mecânica Estatística está se disseminando nas mais variadas áreas do
conhecimento. Esses métodos são reconhecidamente bem-sucedidos na descrição
de fenômenos físicos; com eles, propriedades macroscópicas são descritas
quando temos modelos do funcionamento dos constituintes microscópicos do
sistema em estudo. Uma revisão da área é feita em \cited{Castellano2009}.
Pesquisadores de outras áreas também deram diversas contribuições para a
metodologia de modelagem de agentes, uma revisão desses trabalhos é feita no
livro\cite{Epstein2011}.

No entanto, a ideia de se fazer modelos físicos capazes de descrever
o comportamento social humano  é bem antiga. Já em meados do século
$XIX$ o astrônomo Quetelet cunhou o termo \textit{Physique Sociale} para
designar seus próprios esforços no desenvolvimento de modelos matemáticos
capazes de descrever diferentes aspectos sociais\cite{Stewart1950}. Uma das
contribuições mais importantes dadas por ele foi a aplicação de conceitos
e métodos de probabilidade e estatística, que na sua época eram usados
principalmente na análise de dados astronômicos, em dados coletados da
sociedade (por exemplo taxa de mortalidade). Além disso, é notável que
os conceitos e os métodos matemáticos da probabilidade e estatística
foram desenvolvidos por diversos matemáticos (como Pascal, Fermat, Laplace,
entre outros grandes) tendo como laboratório o estudo dos jogos de azar. Em
seu trabalho, Quetelet introduziu o conceito do \textbf{homem médio}
(\textit{l'homme moyen}), que seria a representação matemática do homem a
partir de variáveis que seguem a distribuição gaussiana. A introdução,
dada por Quetelet, de conceitos estatísticos no contexto social chegou ao
conhecimento de Maxwell e lhe serviu de inspiração na construção das
fundações da Mecânica Estatística \cite{Finberg1992}.

Atualmente, técnicas modernas empregadas em neurociência, como imagens
funcionais de ressonância magnética (\textbf{fMRI}), permitem avaliar as
respostas do cérebro de indivíduos quando esses são submetidos a variados
tipos de estímulos e interações. Além disso, no âmbito da psicologia e
sociologia, a quantidade crescente de pesquisas envolvendo dados numéricos,
como por exemplo pesquisas de opinião, fornecem uma oportunidade para
modelarmos alguns aspectos da sociedade.

Nessa tese, iremos modelar o comportamento do aprendizado moral de pessoas,
através de um modelo de agentes Bayesianos adaptativos interagentes, baseado
no modelo proposto em 2011 por Caticha e Vicente\cited{Caticha2011a}. A
nossa modelagem é fortemente baseada em evidências experimentais. Usaremos
o arcabouço teórico e evidências empíricas fornecidas pela Teoria dos
Fundamentos Morais, que foi proposta e desenvolvida por pelo psicólogo
social Jonathan Haidt juntamente com seus colaboradores em uma série de
trabalhos.  Essa teoria propõe que seres humanos fazem julgamentos morais
de forma predominantemente intuitiva e usando um conjunto de pelo menos cinco
intuições, dimensões ou fundamentos morais, sendo esses: \textit{justiça /
trapaça}, \textit{cuidado / violência}; \textit{lealdade / traição};
\textit{pureza / degradação}; \textit{autoridade / subversão}. Além disso,
a aderência a esses fundamentos depende, em média, da ideologia política
do indivíduo.

A organização do texto principal se dá na seguinte forma: primeiramente
fazemos uma revisão da literatura de ciências sociais de diversas teorias
e resultados experimentais relacionados com o aprendizado moral; em seguida
faremos a descrição do nosso modelo. Por fim, apresentamos e comentamos
alguns resultados de simulação que estão diretamente relacionados com
evidências experimentais. 

