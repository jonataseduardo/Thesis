\chapter{Análise multi cultural}

Uma das perguntas mais importantes que podem ser feita é o quão geral é  a teoria
de aprendizado moral que desenvolvemos durante a tese, se ela só é valida para
os Estados Unidos da América, ou se é geral o suficiente para descrever o
aprendizado moral de sociedades em geral.

Esse tema começou a ser tratado após a conclusão da tese, quando obtivemos o
conjunto de dados completos do questionário da Teoria dos Fundamento
Morais\citep{Quest}.
Como foi visto no apêndice~\ref{sec:rho_delta}, existe uma equivalência
entre a modelagem Bayesiana do aprendizado moral com a modelagem proposta
por Caticha e Vicente\citep{Caticha2011a}. O modelo proposto por Caticha e
vicente tem a vantagem de ser um sistema Hamiltoniano, permitindo assim, que
seja feito de forma mais fácil uma analise de campo médio.

\section{Campo Médio}

A aproximação de campo médio é feita ao maximizarmos a entropia
relativa entre uma distribuição de probabilidade do tipo $P_{cm}=\prod_i
Q_i(\bm \omega_i)$ em relação a distribuição de Gibbs\cite{Opper2001},
$P_G \propto \exp -\alpha \mc{H}$, onde a entropia relativa entre as duas
distribuições é dada por,
\begin{align}
    S\left[P_{cm}||P_G\right] &= - \int \prod_i d \mu \left(\bm \omega_i\right)
    \left\{ P_{cm}\log\frac{P_{cm}}{P_G} - \lambda\left(P_{cm} -1\right)\right\}
    \nn
    &= -\int d \mu \left(\bm \omega_i\right) Q_i\log Q_i +
        \lambda Q_i, \nonumber \\
        &\quad  - \alpha\sum_{\left(i,j\right)}\int d \mu
        \left(\bm \omega_i\right) d \mu
\left(\bm \omega_j\right)Q_iQ_jV_{ij} + cte,
\end{align}
onde $d\mu(\bm \omega)$ é o elemento de área de uma hiperesfera. A maximização
da entropia acontece quando $\frac{\delta S\left[Q||P_G\right]}{\delta Q_i}
= 0$, de onde segue\footnote{No caso do modelo de Ising essa procedimento
resulta na aproximação de campo médio de Curie-Weiss\citep{Opper2001}},
\begin{equation}
Q_i \propto \exp \left( -\alpha \sum_{j \in viz\left(i\right)} \int d
        \mu \left(\bm{J}_j\right) Q_jV_{ij} \right)\nonumber.
\end{equation}

Devido ao formato do Hamiltoniano podemos definir os seguintes
parâmetros de ordem 
\begin{align}
m &= \int d \mu \left(\bm \omega_j\right)
Q_j\bm \omega_j\cdot \mc Z \nonumber \\
r &= \int d \mu \left(\bm \omega_j\right)
Q_j|\bm \omega_j\cdot\mc Z |\nonumber
\end{align}
de forma que a Hamiltoniana de campo  será dada por
\[
\int d \mu \left(\bm \omega_j\right) Q_jV_{ij} = -\frac{1+\delta}{2}h_i m
+\frac{1-\delta}{2}|h_i| r 
\]

Com isso, a distribuição de probabilidade pela aproximação de campo
médio será dada por
\begin{align}
    P_{cm}\left(\bm \omega |\alpha,\delta, \{m,r\}\right) 
    &= \prod_i\frac{1}{Z} \exp \alpha 
    \left( -\frac{1+\delta}{2}h_i m +\frac{1-\delta}{2}|h_i| r\right)\nn
    &= \prod_i\frac{1}{Z} B(h_i|\delta,k\alpha,m,r)
\end{align}
sendo obtido a partir da solução das seguintes equações
auto consistentes
\begin{align}
m &= \frac{1}{Z}\int d \mu\left(\bm \omega\right)
         B(h|\delta,k\alpha,m,r)h\\ 
r &= \frac{1}{Z}\int d \mu\left(\bm \omega\right)
          B(h|\delta,k\alpha,m,r)|h|\\
Z &=  \int d \mu\left(\bm \omega\right) 
            B(h|\delta,k\alpha,m,r)
\end{align}
Considerando a direção do \textit{Zeitgeist} como o eixo de simetria 
\begin{equation}
    P\left(h|\alpha,\delta, \{m,r\}\right) = \int\mu(\bm \omega)
    \delta(\mc Z \cdot \bm \omega - h ) 
    Q\left(\bm \omega |\alpha,\delta, \{m,r\}\right) 
\end{equation}

Experimentalmente estamos interessado na região de parâmetros onde a população
se organiza com grande probabilidade na direção do \textit{Zeitgeist}, ou seja,
quando $h_i>0$, para quase todo agente $i$. Nessa situação, temos que $m\approx
r$, e com isso a distribuição de campo médio será dada por, 
\begin{equation}
    P(h|\delta, k \alpha,\{m\}) \approx
    \frac{\gamma^2}{2} \left(1 - h^2\right) e^{-\gamma (1- h )},
    \label{eq:pgamma}
\end{equation}
onde $\gamma = m \delta k \alpha$.

\subsection{Estimação de máxima verossimilhança} % (fold)
\label{sub:maxver}

Considerando um conjunto de dados de opinião $D= \{h_1,\ldots,h_n\}$
independentes e identicamente distribuídos a partir da probabilidade de campo
médio \eqref{eq:pgamma}, teremos que o logaritmo da verossimilhança para
esse conjunto de dados será,
\begin{equation}
    L(\gamma) = \sum_{i=1}^n P(h_i|\delta, k \alpha,\{m\}) 
        = \sum_{i=1}^n \log P(h_i|\gamma).
\end{equation}
Com isso, podemos estimar o valor do parâmetro $\gamma$ que maximiza a
probabilidade da amostra ter sido sorteada,
\begin{align}
    \frac{d L(\gamma)}{d \gamma} = \sum_i ( -1 + \frac{2}{\gamma} + h_i ) = 0.
\end{align}

Logo, o valor do parâmetro $\gamma$ que maximiza a verossimilhança é dado por,
\begin{align}
    \frac{2}{\gamma} &= 1- \frac{1}{n}\sum_i h_i, \nn
                     &=  1 - \overline m.
    \label{eq:gm}
\end{align}

Assumindo que o número de amostras é muito grande ($n\rightarrow \infty$) pela
lei dos grandes números\cite{DeGroot1989,Wassermann2003} a média amostral de uma variável
aleatória tende à esperança dessa variável, com isso, temos que nesse limite,
$\overline m \rightarrow m $. Lembrando que $\gamma = m \delta k \alpha$,
obtemos a partir do estimador de máxima verosimilhança em \eqref{eq:gm} que
\begin{align}
    m(1-m) = \frac{2}{k\alpha\delta}.
\end{align}
Como assumimos que $m$ é estritamente positivo, temos que a opinião média de um
agente em função do número médio de vizinhos, pressão social e tendencia
corroborativa na aproximação de campo médio quando $m\approx r$ é 
\begin{equation}
    m(k\alpha,\delta) = \frac{1}{2} + \sqrt{\frac{1}{4} +\frac{2}{k\alpha\delta}}.
\end{equation}

De fato, a qualidade da aproximação de campo médio para a regressão
dos histogramas de opinião experimentais pode ser verificada pela
figura~\ref{fig:mf_usa}, onde, em amarelo, é apresentado os histograma das
opiniões morais de 113 mil respondentes dos Estados Unidos da América com
diferentes ideologias políticas. Sendo as linhas azuis as distribuições
de campo médio estimada a partir da maximização da verossimilhança.
 
\begin{figure}
\begin{center}
\includegraphics[width=\textwidth]{Figures/mf_usa}
\end{center}
\caption{Em amarelo é apresentado os histograma das opiniões morais para
grupos de pessoas com diferentes ideologias políticas 113 mil respondentes dos
Estados Unidos da América. As linhas azuis são as distribuição
de probabilidade de campo médio estimada a partir da maximização da
verossimilhança.
} 
\label{fig:mf_usa}
\end{figure}
                 
% subsection Estimação de máxima verossimilhaça (end)

\section{Comparação com \textit{Thigtness/Loseness}}



\begin{figure}
\begin{center}
\includegraphics[width=\textwidth]{Figures/mf_br_jp}
\end{center}
\caption{Histogramas de opiniões de respondentes do Brasil (figuras de cima)
    e do Japão (figuras inferiores), para diferentes ideologias
    políticas. Percebesse que existe uma maior dispersão nas opiniões dos
    respondentes Brasileiros em relação  aos Japoneses com a mesma ideologia
    política.
} 
\label{fig:}
\end{figure}

\begin{figure}
\begin{center}
\includegraphics[width=\textwidth]{Figures/mag_tight}
\end{center}
\caption{Opinião média dos respondentes $m$ por ideologia política de cada pais
em relação ao índice \textit{Tightness/Loseness}. Ao lado da legenda das
ideologias é apresentado a covariância entre a opinião média e índice
\textit{Tightness/Loseness}. }
\label{fig:m_tl}
\end{figure}

\begin{figure}
\begin{center}
\includegraphics[width=\textwidth]{Figures/pro_delta_pa}
\end{center}
\caption{Na figura acima é apresentado a tendência corroborativa em função
da filiação política. Na figura abaixo os pontos azuis representam 
proporção das tendencias corroborativas entre pessoas de ideologias
politicas consecutivas, na escala liberal conservador, para cada pais. Os
pontos vermelhos são as medianas das proporções das tendências corroborativas.
} 
\label{fig:}
\end{figure}

\begin{figure}
\begin{center}
\includegraphics[width=\textwidth]{Figures/beta_tight}
\end{center}
\caption{Relação entre a pressão social do modelo de agentes e o \textit{tightness
score}. As barras de erros foram calculadas a partir de $95\%$ de confiança do
\textit{bootstrap}. A linha vermelha representa o melhor ajuste linear
ponderado pelo erros dos pontos, onde a região em azul representa o intervalo de
de $68\%$ de confiança para essa regressão. A região em cinza é o intervalo de
predição de $68\%$ da regressão linear. Observa-se claramente que a pressão
social medida pela teoria cresce com o \textit{tightness score} do pais, sendo
que as duas medidas apresentão uma correlação de 0.65.}
\label{fig:}
\end{figure}

